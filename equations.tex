\documentclass{beamer}
\mode<presentation>
\usepackage{amsmath}
\usepackage{mathrsfs,amsmath}
\usepackage{amssymb}
%\usepackage{advdate}
\usepackage{adjustbox}
\usepackage{subcaption}
\usepackage{mathtools}
\usepackage{enumitem}
\usepackage{multicol}
\usepackage{listings}
\usepackage{url}
\def\UrlBreaks{\do\/\do-}
\usetheme{Madrid}
\usecolortheme{beaver}
\setbeamertemplate{footline}
{
  \leavevmode%
  \hbox{%
  \begin{beamercolorbox}[wd=\paperwidth,ht=2.25ex,dp=1ex,right]{author in head/foot}%
    \insertframenumber{} / \inserttotalframenumber\hspace*{2ex} 
  \end{beamercolorbox}}%
  \vskip0pt%
}
\setbeamertemplate{navigation symbols}{}

\providecommand{\nCr}[2]{\,^{#1}C_{#2}} % nCr
\providecommand{\nPr}[2]{\,^{#1}P_{#2}} % nPr
\providecommand{\mbf}{\mathbf}
\providecommand{\pr}[1]{\ensuremath{\Pr\left(#1\right)}}
\providecommand{\qfunc}[1]{\ensuremath{Q\left(#1\right)}}
\providecommand{\sbrak}[1]{\ensuremath{{}\left[#1\right]}}
\providecommand{\lsbrak}[1]{\ensuremath{{}\left[#1\right.}}
\providecommand{\rsbrak}[1]{\ensuremath{{}\left.#1\right]}}
\providecommand{\brak}[1]{\ensuremath{\left(#1\right)}}
\providecommand{\lbrak}[1]{\ensuremath{\left(#1\right.}}
\providecommand{\rbrak}[1]{\ensuremath{\left.#1\right)}}
\providecommand{\cbrak}[1]{\ensuremath{\left\{#1\right\}}}
\providecommand{\lcbrak}[1]{\ensuremath{\left\{#1\right.}}
\providecommand{\rcbrak}[1]{\ensuremath{\left.#1\right\}}}
\theoremstyle{remark}
\newtheorem{rem}{Remark}
\newcommand{\sgn}{\mathop{\mathrm{sgn}}}
\providecommand{\abs}[1]{\left\vert#1\right\vert}
\providecommand{\res}[1]{\Res\displaylimits_{#1}} 
\providecommand{\norm}[1]{\lVert#1\rVert}
\providecommand{\mtx}[1]{\mathbf{#1}}
\providecommand{\mean}[1]{E\left[ #1 \right]}
\providecommand{\fourier}{\overset{\mathcal{F}}{ \rightleftharpoons}}
%\providecommand{\hilbert}{\overset{\mathcal{H}}{ \rightleftharpoons}}
\providecommand{\system}{\overset{\mathcal{H}}{ \longleftrightarrow}}
	%\newcommand{\solution}[2]{\textbf{Solution:}{#1}}
%\newcommand{\solution}{\noindent \textbf{Solution: }}
\providecommand{\dec}[2]{\ensuremath{\overset{#1}{\underset{#2}{\gtrless}}}}
\newcommand{\myvec}[1]{\ensuremath{\begin{pmatrix}#1\end{pmatrix}}}
\let\vec\mathbf

\begin{document}

\section*{Outline}
\begin{frame}
\tableofcontents
\end{frame}

\section{Equation 1}
\begin{frame}
\frametitle{Equation 1}
\begin{align*}
 \hat{x}_m[v] ={\mathcal{F}}(u_m[v])=\frac{1}{\sqrt{2}}\{sgn(Re\{u_m[v]\}) +         j.sgn(Im\{u_m[v]\})\}
\end{align*}
\newline
with the signum function $sgn\{a\} = \pm{1}$ for $\mathbb{R} \ni a \gtrless 0.$ 
\end{frame}
 
\section{Equation 2}
\begin{frame}
\frametitle{Equation 2}
 \begin{equation*}
 \vec{w}_{m,i}[v] = \vec{w}^{(CM)}_{m,i}[v] + \vec{w}^{(DD)}_{m,i}[v]
 \end{equation*}
 $\vec{CM}$ Algorithm -- Concurrent Constant ModulusAlgorithm
 \newline
 $\vec{DD}$ Algorithm -- Decision Directed Algorithm
 \newline
 \newline
 where $\vec{w}^{(CM)}_{m,i}[v]$ will be updated by a CM algorithm
 \newline
 $\vec{w}^{(DD)}_{m,i}[v]$ is adjusted in DD mode, with $m \in \{1, 2, . . . 40\}$
being the subcarrier index
 \end{frame}
 

\section{Equation 3}
\begin{frame}
\frametitle{Equation 3}
 \begin{equation*}
 \nu_m[v] =\sum_{i=0}^{2}\vec{w}^{H}_{m,i}[v] \vec{y}_{m,i}[v]
 \end{equation*}
where $\vec{y}_{m,i}[v]$ is a tap-delay-line vector containing a data
window of the polyphase signal $\vec{y}_{m,i}[v]$ in Fig.8,such that
\begin{equation*}
\vec{y}_{m,i}[v] = \myvec{y_{m,i}[v]\\y_{m,i}[v -1]\\.\\.\\.\\y_{m,i}[v- L_{m,i} + 1]}
\end{equation*}
 \end{frame}
 
\section{Equation 4}
\begin{frame}
\frametitle{Equation 4}
 If we neglect carrier frequency and phase offsets, then the
subcarrier output is given by
  \begin{equation*}
 \hat{x}_m[v] ={\mathcal{F}}(u_m[v])
  \end{equation*}
 \end{frame}
 
 \section{Equation 5}
\begin{frame}
\frametitle{Equation 5}
 \begin{equation*}
\vec{w}^{(CM)}_{m,i}[v+1] = \vec{w}^{(CM)}_{m,i}[v] + \Delta{\vec{w}^{(CM)}_{m,i}[v]}\vec{y}_{m,i}[v]
 \end{equation*}
 \end{frame}
  
   \section{Equation 6}
\begin{frame}
\frametitle{Equation 6}
 \begin{equation*}
\Delta{\vec{w}^{(CM)}_{m,i}[v]} = \mu{_{CM}}(1 - \mid{{\nu_m[v]}^2}){\nu^*_m[v]}\vec{y}_{m,i}[v]
\end{equation*}
  \end{frame}
  
   \section{Equation 7}
\begin{frame}
\frametitle{Equation 7}
 \begin{equation*}
{\nu^{(CM)}_m[v]} = \sum_{i=0}^{2} (\vec{w}^{(CM)}_{m,i}[v] + \Delta{\vec{w}^{(CM)}_{m,i}[v]})^{H}\vec{y}_{m,i}[v]
\end{equation*}
  \end{frame}
  
   \section{Equation 8}
\begin{frame}
\frametitle{Equation 8}
 \begin{equation*}
\vec{w}^{(DD)}_{m,i}[v+1] = 
 \end{equation*}
 \begin{equation*}
 \vec{w}^{(DD)}_{m,i}[v] + \mu{_{DD}}.\delta(\hat{x}_m[v]-{\mathcal{F}}(u_m[v])).({\mathcal{F}}(u_m[v]) - {\nu_m[v]})^*\vec{y}_{m,i}[v]
 \end{equation*}
 \newline
 \newline
 where,
\begin{align*}
\delta(a)=  \begin{dcases}
        1 & a=0 \\
        0 & a\neq 0 \\
    \end{dcases}
\end{align*} 
\end{frame}

\end{document}